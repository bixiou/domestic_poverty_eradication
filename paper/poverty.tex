%%%%%%%%%%%%%%%%%%%%%%%%%%%%%%%%%%%%%%%%
%%%%% NATURE CLIMATE CHANGE FORMAT %%%%%
%%%%%%%%%%%%%%%%%%%%%%%%%%%%%%%%%%%%%%%%
%% Comment "% WPcomment" lines, uncomment "% NCCcomment" lines as well as the lines below, replace all citet/citep by cite

% \documentclass{nature}
% \usepackage{amsmath}
% \usepackage{amssymb}
% \usepackage{eurosym}
% % The following allows keeping figures within the text (otherwise nature.cls would ignore them)
% \usepackage{graphicx}
% \makeatletter
% \let\saved@includegraphics\includegraphics
% \AtBeginDocument{\let\includegraphics\saved@includegraphics}
% \renewenvironment*{figure}{\@float{figure}}{\end@float}
% \makeatother

% Nature guidelines (not NCC!)
% Sections can only be used in Articles.  Contributions should be organized in the sequence: title, text, methods, references, Supplementary Information line (if any), acknowledgements, interest declaration, corresponding author line, tables, figure legends.

% No subsubsection nor paragraph

% Spelling must be British English (Oxford English Dictionary)

%Each figure legend should begin with a brief title for the whole figure and continue with a short description of each panel and the symbols used. For contributions with methods sections, legends should not contain any details of methods, or exceed 100 words (fewer than 500 words in total for the whole paper). In contributions without methods sections, legends should be fewer than 300 words (800 words or fewer in total for the whole paper).

% Articles are restricted to 50 references,

% In addition, a cover letter needs to be written with the
% following:
% \begin{enumerate}
%  \item A 100 word or less summary indicating on scientific grounds
% why the paper should be considered for a wide-ranging journal like
% \textsl{Nature} instead of a more narrowly focussed journal.
%  \item A 100 word or less summary aimed at a non-scientific audience,
% written at the level of a national newspaper.  It may be used for
% \textsl{Nature}'s press release or other general publicity.
%  \item The cover letter should state clearly what is included as the
% submission, including number of figures, supporting manuscripts
% and any Supplementary Information (specifying number of items and
% format).
%  \item The cover letter should also state the number of
% words of text in the paper; the number of figures and parts of
% figures (for example, 4 figures, comprising 16 separate panels in
% total); a rough estimate of the desired final size of figures in
% terms of number of pages; and a full current postal address,
% telephone and fax numbers, and current e-mail address.
% \end{enumerate}

% See \textsl{Nature}'s website
% (\texttt{http://www.nature.com/nature/submit/gta/index.html}) for
% complete submission guidelines.

%%%%%%%%%%%%%%%%%%%%%%%%%%%%%%%%
%%%%% WORKING PAPER FORMAT %%%%%
%%%%%%%%%%%%%%%%%%%%%%%%%%%%%%%%
%% Comment "% NCCcomment" lines, uncomment "% WPcomment" lines as well as the lines below
\documentclass[12pt,english]{article}
\usepackage[utf8]{inputenc}
\usepackage{tgpagella} % Palatino text only
\usepackage{mathpazo}  % Palatino math & text
\usepackage[left=1.5in,right=1.5in,top=1.5in,bottom=1.5in]{geometry}
% \linespread{1.5}
\usepackage[super,comma,sort]{natbib} % WPcomment
% \usepackage[round,sort&compress]{natbib} % NCCcomment
\usepackage{url} % [hyphens]
\usepackage[hyperpageref]{backref} % back references biblio. Needs latexmk at compilation.
\usepackage[pagebackref]{hyperref}
% \usepackage{multibib} % incompatible with backref
\hypersetup{
  colorlinks=true, % breaklinks=true,
  urlcolor=purple,    % color of external links
  linkcolor=blue,  % color of toc, list of figs etc.
  citecolor=violet,   % color of links to bibliography
}
\usepackage{bm}
\usepackage{indentfirst}
\usepackage{tocbibind}
\setcitestyle{aysep={}} 
\usepackage{amsmath}
\usepackage{tcolorbox}
\usepackage{amssymb}
\usepackage{eurosym}
\usepackage{amsfonts}
\usepackage{enumerate}
\usepackage{babel}
\usepackage{graphicx}
\usepackage{caption}
\usepackage{supertabular}
\usepackage{tabularx}
\usepackage{float}
\usepackage{dsfont}
\usepackage{fancyvrb}
\usepackage{verbatim}
\usepackage{enumitem}
\usepackage{setspace}
\usepackage{comment}
\usepackage{subcaption}
\usepackage{tikz}
\usepackage{gensymb}
\usepackage{textcomp}

\usepackage{tabulary}
\usepackage{tabularx}
\usepackage{booktabs}
\usepackage{fullpage}
\usepackage{morefloats}
\usepackage{makecell}
\usepackage{lscape}
\usepackage{pdflscape}
\usepackage{longtable}
\usepackage{rotating}
\usepackage{fancyhdr}
\usepackage{tocloft}
\usepackage{titletoc}
\usepackage[export]{adjustbox}
\usepackage[anythingbreaks]{breakurl} % for links
\usepackage{multicol}
\newsavebox\ltmcbox % For net gain table over two columns
%\usepackage[nomarkers,figuresonly]{endfloat} % Figures at the end
%\usepackage[section,below]{placeins} % Floats placed in the section they appear in.
\renewcommand{\floatpagefraction}{.99}
\newenvironment{stretchpars}{\par\setlength{\parfillskip}{0pt}}{\par} % to justify a line

% % Getting landscape page and page number/footer on bottom of page (instead of to the left)
% \fancypagestyle{mylandscape}{
% \fancyhf{} %Clears the header/footer
% \fancyfoot{% Footer
% \makebox[\textwidth][r]{% Right
%   \rlap{\hspace{1.5cm}% Push out of margin by \footskip
%     \smash{% Remove vertical height
%       \raisebox{13.6cm}{% Raise vertically
%         \rotatebox{90}{\thepage}}}}}}% Rotate counter-clockwise
% \renewcommand{\headrulewidth}{0pt}% No header rule
% \renewcommand{\footrulewidth}{0pt}% No footer rule
% }

% \fancypagestyle{page_left}{%
% 	\renewcommand{\headrulewidth}{0pt}
%   \fancyhf{}
%   \fancyfoot[OC]{%
%       \begin{tikzpicture}[remember picture,overlay]
%           \node[xshift=1cm] (number) at (current page.west) {\thepage};
%       \end{tikzpicture}
%   }%
% }
% \renewcommand{\thesubfigure}{\Alph{subfigure}}

% \newcites{App}{Appendix References}

% \captionsetup[table]{skip=-10pt}
% \begin{document}

% \maketitle

% \clearpage
% % \startcontents
% % \printcontents{ }{1}{\section{\contentsname}}
% % \clearpage
% \section{Introduction\label{sec:intro}}

% % \clearpage
% \renewcommand{\bibsection}{\section{\refname}}
% \bibliographystyle{naturemag}
% \bibliography{global_tax_attitudes}
% % \stopcontents

% \end{document}


\title{Shortfall of Domestic Resources to Eradicate Extreme Poverty} 

\author{Adrien Fabre$^{1,2}$} % WPcomment
\author{Adrien Fabre\footnote{CNRS, CIRED. E-mail: adrien.fabre@cnrs.fr.}
I thank Michalis Moatsos for his help in using his data. I thank Thomas Goumont and Elise Thai for assistance. I declare that I also serve as president of Global Redistribution Advocates.} % NCCcomment

\date{\today} % NCCcomment

\begin{document}

\maketitle

\begin{center}
{\textbf{\href{https://github.com/bixiou/domestic_poverty_eradication/raw/main/paper/poverty.pdf}{Link to most recent version}}}
\end{center}


% WPcomment
% \begin{affiliations}
% \item CNRS
% \item CIRED
% \end{affiliations}

% \begin{small} % NCCcomment
\begin{abstract}

\end{abstract}

% TODO!
\textbf{JEL codes:} 
\textbf{Keywords:} 

\tableofcontents

\onehalfspacing % NCCcomment

%\clearpage

\section{Introduction}% NCCcomment


\paragraph{Literature} 



\section{Results}
\subsection{Data}\label{subsec:data}
% => Why using PIP compared to alternatives (WIID adjusted to GDP; (GCIP stops in 2013; WID only covers 38 countries with post-tax data))? Most recent data and best estimation of poor's consumption.
% Not clear which is more accurate between survey and national accounts (Deaton 05), especially in low-income countries, as agricultural production is indirectly measured in national accounts. Also, in Africa, mean conso is similar between national accounts and surveys (Deaton 05). Also, WIID adjust to GDP PPP but I should adjust to national conso, not GDP. Martinez (22) shows that autocracies overestimate GDP growth by 35%. 
% Prydz et al. (22) argue that NAS are more accurate but show that the discrepancy between survey mean consumption and NAS HFCE is not that large: 22%, and 13.7% for low-income countries. HFCE is broader (and encompasses spending of non-profit entities like NGOs). => Also, if one scales up the conso distribution, one should also scale up the poverty line (as it is based on conso surveys): if one does that, poverty rates are on average the same, they change due to variation in survey/NAS discrepancies. 
% I can adjust to HFCE as robustness check https://data.worldbank.org/indicator/NE.CON.PRVT.PP.KD with the assumption that the extra money is held by the rich (>13$/day), 
The percentiles of each country's income (or consumption) are estimated by the Poverty and Inequality Platform (PIP) of the World Bank (ex-PovcalNet). This data is based on purchasing power parity (PPP) and given in constant 2017 \$. PIP aggregates the most recent household surveys (60\% of countries were surveyed between 2018 and 2021). 

In low-income countries (those of greatest interest to us), PIP provides data on the per capita \textit{consumption} (rather than income). Thereby, the data does not capture services procured by the government. Another potential concern with household surveys is that the aggregate (national) consumption they imply is generally lower than the one estimated in national accounts.\cite{deaton_measuring_2005,prydz_disparities_2022} This discrepancy comes from measurement errors on both sides: on the one hand, household surveys suffer from underreporting of top incomes and large expenditures; on the other hand, national accounts do not properly account for informal work %and auto-consumption, 
and tend to inflate agricultural output.\cite{aangrist_why_2021} 
Furthermore, autoritarian countries have been shown to produce inflated GDP statistics, except for countries below the GDP threshold of eligibility for preferential loans by the World Bank.\cite{martinez_how_2022} % international development association
While the ratio of Household Final Consumption Expenditures (HFCE) from national accounts is 44\% greater than the aggregate value from household surveys, the ``discrepancy ratio'' is largest for middle-income countries, and is only 12\% for low-income countries. 
Because household surveys are best suited to estimate consumption by the poorest, I use unadjusted PIP data in our baseline. 

As a robustness check, I also re-derive our main results after adjusting aggregate consumption by the discrepancy ratio (computed using World Bank data). In line with the literature,\cite{lakner_global_2013,anand_chapter_2015} I impute the extra consumption to the top percentile. I do not perform the rescaling on the 15\% of countries with HFCE lower than its aggregate consumption from PIP, and I assume a discrepancy ratio of +12\% for the 20\% of countries lacking data on HFCE. 

As is common in this literature,\cite{karver_mdgs_2012,hellebrandt_future_2015,bicaba_can_2017} my baseline assumes ``balanced growth'', meaning that each percentile grows at the same rate between the country's survey year and 2030. 
I rescale incomes by the observed growth of GDP p.c. (in PPP) up to 2022 (using World Bank data) and by different methods for the 2022--2030 period. 
These methods include: extending the 2014--2019 growth trend (which excludes COVID years); extending the trend for growing countries and assuming no growth when GDP p.c. has contracted between 2014 and 2019; assuming a constant growth (of either 0\%, 3\%, 4.5\%, 6\%, or 7\%); using IMF forecasts\cite{imf_world_2023} (extended up to 2030 by replicating the 2026--2028 forecasted growth in 2028--2030); projecting future growth using a quadratic model that predicts the 2011--2019 growth based on the 1991-2011 growth (then applied to 2022--2030 using the 2002--2022 growth). I deviate from this two-step procedure assess the original SDG goal, as I assume a constant growth of 7\% starting in 2015.



\subsection{The effect of balanced growth}
% - Poverty rates and gaps with different growth scenarios: Table 1
% This assumption is the same than PIP's https://pip.worldbank.org/about

To estimate global poverty rates, the World Bank scales up the percentiles measured in household surveys by the country's GDP growth between the survey year and the year of interest. I project global poverty rates and poverty gaps in 2030 using the same assumption of balanced growth, for a range of growth scenarios (Table \ref{tab:poverty_rates}).

\subsection{Antipoverty caps}
% - Cap to eradicate poverty: minimum of 2.7\$/day in MDG with 7\% growth, 1.74 with trend growth (1\%): Figure 1
% - would have worked well if it had been 7\% since 2016

\subsection{Antipoverty taxes}
% By default, 6\% (7?) growth starting in 2023
% - Tax rate to eradicate poverty: above 7, 18: Figure 2
% - Even worse when considering Moatsos' BCS line

\subsection{The credible potential of domestic redistribution}
% - Demogrant that can be funded with a given national tax: Figure 3

\subsection{The potential of global redistribution}
% - One example of global redistribution (incl. largest recipients/contributors)
% - Inequality indicators BAU / national / global redistr: Table 2


\section{Discussion} 

% \begin{methods}  % WPcomment
  \begin{small} % NCCcomment
%Put methods in here.  If you are going to subsection it, use \subsection commands.  Methods section should be less than 800 words and if it is less than 200 words, it can be incorporated into the main text.
\section*{\normalsize Methods}\label{sec:methods} % NCCcomment
\addcontentsline{toc}{section}{\nameref{sec:methods}}
% \subsection*{\small Data quality.} % WPcomment % TODO attrition analysis
\paragraph{\small Data quality.} % NCCcomment


\section*{\normalsize Data and code availability}

All data and code of as well as figures of the paper are available on \href{https://github.com/bixiou/domestic_poverty_eradication}{github.com/bixiou/domestic\_poverty\_eradication}. 

% \end{methods} % WPcomment
\end{small}  % NCCcomment

% \bibliographystyle{naturemag_noURL} % nature class works only with style naturemag or naturemag_noURL, and naturemag bugs if there are certain URLs (like .pdf). Also, nature class only works with \cite, not \citet or \citep.  % WPcomment
\renewcommand{\url}[1]{\href{#1}{Link}} % NCCcomment
\bibliographystyle{plainnaturl_clean} % NCCcomment
\bibliography{poverty}

\appendix % NCCcomment
\renewcommand{\thetable}{A\arabic{table}}
\renewcommand{\thefigure}{A\arabic{figure}}
\setcounter{figure}{0}
\setcounter{table}{0}

% \input{app} 

\clearpage
\listoftables
\listoffigures
% WPcomment
%% Here is the endmatter stuff: Supplementary Info, etc.
%% Use \item's to separate, default label is "Acknowledgements"
% \begin{addendum} % 177 words
%  \item I are grateful 
%  \item[Competing Interests] The authors declare that they have no
% competing interests.
% \item[JEL codes] 
% \item[Keywords] 
%  \item[Correspondence] Correspondence and requests for materials
% should be addressed to Adrien Fabre~(email: fabre.adri1@gmail.com).
% \end{addendum}


\end{document}
