
\documentclass[12pt,english]{article}
\usepackage[utf8]{inputenc}
\usepackage[left=1.5in,right=1.5in,top=1.5in,bottom=1.5in]{geometry}
\usepackage{bm}
\usepackage{amsmath}
\usepackage{amssymb}
\usepackage{indentfirst}
\usepackage[hyperpageref]{backref} % back references biblio
\usepackage{tocbibind}
\usepackage[round,sort&compress]{natbib}
\setcitestyle{aysep={}} 
\usepackage{amsfonts}
\usepackage{enumerate}
\usepackage{babel}
\usepackage{caption}
\usepackage{supertabular}
\usepackage{tabularx}
\usepackage{float}
\usepackage{dsfont}
\usepackage{fancyvrb}
\usepackage{verbatim}
\usepackage[hyphens]{url}
\usepackage{hyperref}
\usepackage[shortlabels]{enumitem}
\usepackage{setspace}
\usepackage{comment}
\usepackage{subcaption}
\usepackage{graphicx}
\usepackage{tikz}
\usetikzlibrary{shapes,backgrounds,positioning}
\usepackage{gensymb}
\usepackage{eurosym}
\usepackage{textcomp}
\usepackage{color,soul}

\usepackage{multicol}
\usepackage{changepage}
\usepackage{enumitem}


\usepackage{tabulary}
\usepackage{tabularx}
\usepackage{booktabs}
\usepackage{fullpage}
\usepackage{morefloats}
% \usepackage[utf8]{inputenc}
% \usepackage{bm}
% \usepackage{indentfirst}
% \usepackage{tocbibind}
% \usepackage{enumerate}
\usepackage{makecell}
\usepackage{multirow}
\usepackage{ulem}
\usepackage{lscape}
\usepackage{pdflscape}
\usepackage{longtable}
\usepackage{rotating}
\usepackage{fancyhdr}
\usepackage{tocloft}
\usepackage{multibib}
\usepackage{titletoc}
\usepackage[export]{adjustbox}
\usepackage[anythingbreaks]{breakurl} % for links
%\usepackage[nomarkers,figuresonly]{endfloat} % Figures at the end
\hypersetup{
  colorlinks=true, % breaklinks=true,
  urlcolor=blue, % color of external links
  linkcolor=blue,  % color of toc, list of figs etc.
  citecolor=violet,   % color of links to bibliography
}   

\title{Shortfall of Domestic Resources to Eradicate Extreme Poverty by 2030 ~\\ ~\\ \textbf{Responses to the Editor and Reviewers}}

\date{\today}
\begin{document}
	
\maketitle

\paragraph*{Editor's comments}

\textit{Dear Dr. Fabre,}

\textit{I refer to your paper entitled  "Shortfall of Domestic Resources to Eradicate Extreme Poverty by 2030" which you submitted to the Journal of Development Studies, and regret to inform you that we are unable to publish the paper in its present form.}

\textit{However, should you wish to revise it, taking fully into account the comments made by the referees and providing an explanation of the ways in which they have been dealt with, and justifying those you choose not to address, we would be prepared to consider it again, although you will understand that no undertaking to publish such a version can, at this stage, be made.}~\\

I am very grateful for the opportunity to revise the paper and I will address all the comments made by the referees. I will provide a detailed response to each comment and explain how I have dealt with them. ~\\

\textit{We would also ask you to do everything possible to make the paper as succinct and short (maximum 8,000 words with as few tables as feasible) as you can. The Appendix is too long.  A brief Appendix (about 1,000 words) may be included with papers to present important detail such as data sources and definitions and perhaps descriptive statistics.  % TODO?
However, detailed material (such as on data construction where this is involved, additional results and sensitivity analysis) should be provided in a separate file as Supplementary Materials, which ideally would also include data and do files (where applicable and presented clearly with annotation to facilitate replication). Supplementary Materials can be provided in any file format and will be deposited in Figshare with a citable DOI linked to the article.}~\\

I kept the paper succinct: the wordcount is now TODO words. I moved the additional results from Appendix to a Supplementary Material. The code is also included as Supplementary Materials. I have deposited the manuscript and Supplementary Materials in Figshare and added a DOI link to the article. TODO 
~\\ ~\\

\textit{It is JDS policy that authors of accepted papers should make the data and code available (to bona fide researchers), either in a website or by committing to provide on request.  We would be grateful if you could include a statement in the acknowledgements of how you will make it available.}~\\

For the moment, I have removed the acknowledgements section to respect the double blind review process. However, I have included a Data and code availability section stating that the data and code is freely available on github. It reads as follows: ``All data and code of as well as figures of the paper are available on \href{https://github.com/bixiou/domestic_poverty_eradication}{github.com/bixiou/domestic\_poverty\_eradication}. Many more figures (with varying poverty lines, taxation thresholds, growth scenarios, etc.) are available on the repository. %\href{https://github.com/bixiou/domestic_poverty_eradication/tree/main/figures}{github.com/bixiou/domestic\_poverty\_eradication}. 
Also, any custom figure can be easily produced using this code.''
~\\ ~\\

\textit{Please ensure that your revised version follows our housestyle guides which are available on the site under 'Instructions to Authors'.  Please note that footnotes should appear as endnotes, after the text and before the references.   Pay particular attention to requirements for bibliographic details in references (e.g. include place of publication for books and mimeos, page span for articles and chapters in books).  In addition, please provide the issue number only if each issue of the journal begins on page 1.  In such cases it goes in parentheses:  Journal, 8(1), pp–pp.  Page numbers should always be provided.  Please refer to the following link:   https://www.tandf.co.uk//journals/authors/style/reference/tf\_APA6.pdf}~\\

I have move the footnote as endnote as indicated. %As I didn't succeed in changing the bibliographic style to APA-6 using a LaTeX command, I would need to do it by hand. 
I hope it is fine with you if I wait for an acceptance decision to fix the references.
~\\ ~\\

\textit{The revised documents need to be in editable format and as you have submitted your paper in pdf format that is not possible. Please either convert your paper to Word or upload all the source files.}~\\

The revised submission is in LaTeX format. I have uploaded the source files.
~\\ ~\\

\textit{Please ensure that any Tables, Equations and Figures are submitted in editable format not in image format.}~\\

The Tables are in LaTeX format and the Figures in PDF format, not in image format.
~\\ ~\\


\paragraph*{Reviewer \#1}

\textit{Summary}

\textit{This paper explores the potential fulfilment of SDG1, which focuses on eradicating poverty by 2030. The analysis is structured around three distinct scenarios: (1) poverty eradication achieved exclusively through economic growth, (2) implementation of two extreme tax schedules, and (3) global redistribution to address poverty. }~\\

response
~\\ ~\\

\textit{The topic is highly relevant, as achieving the Sustainable Development Goals (SDGs) remains a pressing global challenge, with poverty eradication at its core. However, as it stands, the paper does not fully meet the standard expected for publication in The Journal of Development Studies. In the following comments, I outline major areas for improvement that I believe will strengthen the paper's contribution and overall impact.}~\\

response
~\\ ~\\ 

\textit{Major comments}

\textit{I encourage the authors to ensure that the text is clear and concise. For instance, terms such as "imputation of national growth rates" and "extreme tax schedules" should be clearly explained for a broader audience. Another example of unclear statements can be found in the introduction (second paragraph). “I fix the rate (at 100\%) and find the required taxation threshold”. From the very beginning of the paper, it is not clear what this sentence means. The first paragraph of Section 3.4 reads, “Figure 2 presents the (additional) tax rate”. What does additional mean? Additional to the antipoverty cap? }~\\

response
~\\ ~\\

\textit{The authors need to better articulate the specific contribution of their work. The paper appears to be an updated application of previous studies, particularly those by Bolch et al. (2022). However, the novelty of the current study is not sufficiently clear. For example, while the authors highlight the imputation of national growth rates and the use of a poverty line aligned with SDG1, these aspects are not sufficiently emphasized or contextualized within the broader literature. How does this paper advance the field beyond being an update? Does it offer new insights into the policy implications of SDG1? Are the results of the scenarios robust to different assumptions about economic growth or redistribution? These points should be explicitly addressed to establish the paper's unique contribution. }~\\

response
~\\ ~\\

\textit{The literature review needs significant improvement. In the last paragraph of the introduction, the authors briefly mention that their work is related to studies on global income distribution but fail to explain how and why. In addition, I wonder how do the scenarios modeled in this paper compare with similar exercises in the literature? Are there methodological or theoretical advancements introduced here? Strengthening the link to prior research will help readers understand the relevance and placement of the study within the broader academic debate. }~\\

response
~\\ ~\\

\textit{While the three scenarios provide a useful framework, they are not adequately justified. Why were these specific scenarios chosen? Are they exhaustive or representative of potential pathways to achieving SDG1? }~\\

response
~\\ ~\\

\textit{For the growth-only scenario, is the assumption of uniform distribution of economic growth realistic? What do historical trends or projected economic conditions say? Assuming that these trends prevail in the upcoming years, how could they affect the results? }~\\

response
~\\ ~\\

\textit{In the global redistribution scenario, how are political feasibility and implementation challenges addressed? These considerations are critical for translating the findings into actionable insights. }~\\

response
~\\ ~\\

\textit{Related to taxation scenarios, I wonder if the antipoverty cap would not generate new poor individuals. If the whole income share up to a certain threshold is completely redistributed to the bottom, all individuals with incomes above that level would automatically have zero income and become poor. }~\\

response
~\\ ~\\

\textit{A thorough discussion of the study's limitations is essential to enhance its credibility and contextualize its findings. This discussion should be integrated into the final sections of the paper to provide a balanced perspective and acknowledge potential constraints. For instance, the scenarios are based on specific assumptions about global trends and policies, which may not account for heterogeneity across countries and regions. A discussion about these contextual factors would help frame the study's conclusions more accurately. The study focuses primarily on economic variables, such as growth, taxation, and redistribution. }~\\

response
~\\ ~\\

\textit{However, non-economic factors like education, healthcare access, and social protection systems also play critical roles in poverty reduction. Acknowledging the exclusion of these dimensions would provide a more nuanced view of the analysis.}~\\

response
~\\ ~\\


\paragraph*{Reviewer \#2}

\textit{This paper examines whether growth and domestic redistribution are sufficient to eradicate extreme poverty by 2030. The authors estimate the parameter of two tax policies that would raise enough revenues to eradicate poverty. In the “antipoverty cap”, the rate is fixed at 100\% to find the required taxation threshold. In the “antipoverty tax”, the threshold is fixed to find the rate needed. In the lowest income countries, extreme poverty is estimated to persist even after strong growth and radical redistribution. The paper concludes by exemplifying international transfers that would eradicate poverty by 2030. }~\\

response
~\\ ~\\

\textit{This paper addresses an issue that is apt and policy relevant. While the findings are mostly in line with the existing literature, I believe that contribution of the paper can be made stronger. I have three main concerns and some recommendations that I hope the author would find useful. }~\\

response
~\\ ~\\
 
\textit{Main Concerns:}

\textit{1.	Limited novelty}
\textit{[page 3] as it stands, the main contribution of this paper is to employ the Bolch et al. (2022) methodology to the most recent data i.e., the period 2018–2021 and to a larger sample of countries by imputing income data using consumption for countries with no income data available. The author claims using a poverty line higher than the one officially used in the first SDG as a contribution, which to me serves more like a robustness check. Altogether, these contributions reflect limited novelty for the Journal of Development Studies. The author needs to substantiate this part. One way to do this is to highlight the difference between the outcomes currently discussed and the outcomes obtained using latest data, but the sample of countries used in Bolch et al (2022) data. This will also give the readers a sense of how credible the imputation procedure is.    }~\\

response
~\\ ~\\

\textit{2.	Absence of a theoretical framework}
This paper employs a methodology like Bolch et al. (2022) to assess which countries have sufficient domestic resources to achieve the first SDG. It is important to briefly discuss this framework in the paper in a separate methodology section as I had to go back to Bolch et al. (2022) to refresh my knowledge about this methodology. }~\\

response
~\\ ~\\

\textit{3.	Narrow scope of the methodology}

\textit{[page 6] the first paragraph starts with “Assuming that each country will continue to grow at the same rate as in the recent past…”, it would be nice if the author considers scenarios of heterogenous growth across countries. As this static framework is not suitable for drawing any causal inference, it would be good if the author mentions the difference between correlational and causal outcomes in the paper while interpreting the outcomes. }~\\

response
~\\ ~\\

\textit{Minor points:}
 
\textit{4.	The structure of the paper}
\textit{This paper currently reads more like a book chapter. It would be necessary to restructure it to be accepted in a journal. For instance, a summary of the key results must be included in the introduction. The key contributions of the paper can be discussed in light of the literature in the introduction.}~\\

response
~\\ ~\\

\textit{5. In the conclusion, the author mentions the case of China's profound government commitment. It is important that the author highlights how feasible such interventions in other developing countries, whether there could be any institutional barriers. Finally, as the author alludes to transfer based on international solidarity as a panacea to eradicate extreme poverty, it would be nice to discuss some policy directives on this matter. How feasible are these interventions based on the transfer of global income to eradicate poverty? This question remains largely unanswered in this paper.}~\\


~\\ ~\\

\clearpage 
\renewcommand{\url}[1]{\href{#1}{Link}} 
% \bibliographystyle{plainnaturl_clean} 
% \bibliography{global_tax_attitudes}

\end{document}
